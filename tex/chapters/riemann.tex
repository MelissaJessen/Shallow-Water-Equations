\section{The Riemann problem}
We will now define the Riemann problem, since it plays a crucial role in the finite volume method.
In the Riemann problem we distinguish between what we call a wet bed and a dry bed. 
A wet bed is the case where the water depth is positive everywhere, whereas a dry bed is the case where the water depth is zero in some cells.
The special Riemann problem where parts of the bed are dry is dealing with the so-called dry fronts or wet/dry fronts, which are challenging to handle numerically.
We will leave these cases for now, and only consider the wet bed problems.

The Riemann problem for the shallow water equations is defined as the initial-value problem (IVP)~\cite{Toro2024}:
\begin{equation}\label{eq:Riemann_problem}
    \begin{aligned}
        \text{PDEs: } &\mathbf{U}_t + {\mathbf{F(U)}}_x = 0, \\
        \text{ICs: } &\mathbf{U}(x, 0) = \begin{cases}
            \mathbf{U_L}, & \text{if  } x < 0, \\
            \mathbf{U_R}, & \text{if  } x > 0.
        \end{cases}
    \end{aligned}
    \end{equation}
Here, the vectors $\mathbf{U}$ and $\mathbf{F(U)}$ in~\eqref{eq:Riemann_problem} are given By
\begin{align}
    \mathbf{U} = \begin{bmatrix}
        h \\ hu \\ hv
    \end{bmatrix}, \quad
    \mathbf{F(U)} = \begin{bmatrix}
        hu \\ hu^2 + \frac{1}{2}gh^2 \\ hvu
    \end{bmatrix},
\end{align}
and the initial conditions $\mathbf{U_L}$ and $\mathbf{U_R}$ are
\begin{align*}
    \mathbf{U_L} = \begin{bmatrix}
        h_L \\ h_L u_L \\ h_L v_L
    \end{bmatrix}, \quad 
    \mathbf{U_R} = \begin{bmatrix}
        h_R \\ h_R u_R \\ h_R v_R
    \end{bmatrix},
\end{align*}
which represents the conditions at time $t = 0$ in the left and right states of $x=0$, respectively.
The function $\mathbf{U}$ is piecewise constant, with a discontinuity at $x=0$.
The Riemann problem is a generalisation of the so-called dam-break problem.
The difference is that in the Riemann problem the particle velocity components, $u_L, u_R, v_L$ and $v_R$, are allowed to be distinct from zero, whereas in the dam-break problem, they must be zero.

The Riemann problem can be solved both exactly and approximately.
There are several approximate Riemann solvers, such as the HLL and Roe solvers, which are based on the approximate solution of the Riemann problem.
We will consider some of these solvers later in the thesis.

\subsection{Waves in the Riemann problem}
To get a better understanding of the flow in shallow water, we provide some very short background information about waves.
In particular the wave structure in the solution of the Riemann problem~\eqref{eq:Riemann_problem}.

In the solution of the Riemann problem~\eqref{eq:Riemann_problem} there are four possible wave patterns outcomes, which are combinations of shock waves and rarefaction waves.
In each case there are three waves, the left and right waves correspond to the one-dimensional SWE, and the middle wave aries from the $y-$momentum equation in~\eqref{eq:Riemann_problem} and is always a shear wave.
The left and right waves are either shock waves or rarefaction waves.
The four possible wave patterns are illustrated in Figure~XX and are as follows:
\begin{enumerate}
    \item Left rarefaction, right shock
    \item Left shock, right rarefaction
    \item Both left and right rarefaction
    \item Both left and right shock
\end{enumerate}
Hence the structure of the solution in general is shown in the figure below.
From the figure we see that the solution consists of three waves, a left wave, a middle wave and a right wave, which together seperate four regions, described by the vector
\begin{align*}
\mathbf{W} = \begin{bmatrix}
    h \\ hu \\ hv
    \end{bmatrix}.
\end{align*}
The four regions are describes by $\mathbf{W}_L$ (left data), $\mathbf{W}_R$ (right data), $\mathbf{W}_{*L}$ and $\mathbf{W}_{*R}$, which both denote star region data.
We are interested in the star region data, since these are the unknowns.
We know that the left and right waves are always either a shock wave or a rarefaction wave, and the middle wave is always a shear wave.
Based on the given initial conditions, we must determine the types of waves.
Second, it is known that across the left and right waves, both $h$ and $u$ change but $v$ remains constant.
Whereas across the middle wave, $v$ changes but $h$ and $u$ remain constant.
%That is, $h$ and $u$ remain constant in the star region.
Thus the water depth and particle velocity are constant in the star region and are denoted by $h_*$ and $u_*$, respectively.


\subsection{Exact Riemann solver}

The exact Riemann solver presented in this section is very efficient and leads to Gudonov methods, that are only slightly more expensive than those based on approximate Riemann solvers~\cite{Toro2001-Shock}.
Important to find the exact solution to the Riemann problem in the early stage of development, before moving on to more complex applications.


In the solution of the Riemann problem~\eqref{eq:Riemann_problem}, there are four possible wave patterns outcomes, which are combinations of shock waves and rarefaction waves.
In each case there are three waves, the left and right waves correspond to the one-dimensional SWE, and the middle wave aries from the $y-$momentum equation in~\eqref{eq:Riemann_problem}.
The left and right waves are either shock waves or rarefaction waves, and the middle wave is always a shear wave.

But for now, we will consider the case where the solution consists of a single non-trivial wave and all other waves are assumed to have zero strenght.
This is enough to solve the Riemann problem as it is always possible to solve the Riemann problem by considering one wave at a time.

We denote the constant values of the water depth and particle velocity in the star region by $h_*$ and $u_*$, respectively.

