\section{SWE in Spherical Coordinates}
Until now we have derived the shallow water equations in cartesian coordinates.
In this section, we will derive the shallow water equations in spherical coordinates.
We will follow the methods used in~\cite{Castro2017}~\cite{Bihlo2022},~\cite{Raymond} and~\cite{Gill_1982}.
To illustrate the spherical coordinates, we will use the latitude and longitude system, visualized in~\autoref{fig:lat-long-earth}.
\begin{figure}[H]
    \centering
    \includegraphics[width=0.5\textwidth]{C:/Users/Matteo/Shallow-Water-Equations/figs/lat-long-earth.jpg}
    \caption{Illustration of latitude and longitude on the planet earth.
    Illustration from~\cite{lat-long-earth}.}\label{fig:lat-long-earth}
\end{figure}
We see that the latitude direction is the north-south component, whereas the longitude direction is the east-west component.
The latitude angle, denoted by $\phi$, goes from $-\frac{\pi}{2}$ at the south pole to $\frac{\pi}{2}$ at the north pole, and the longitude angle, denoted by $\theta$, goes from $0$ at the prime meridian, increasing to the east, to $2\pi$.
The spherical coordinates we use are ($r$, $\theta$, $\phi$), where $r$ is the radius from the center of the sphere, $\theta$ is the longitude angle, and $\phi$ is the latitude angle.
This also means, that any point on the surface of the sphere can be represented by the coordinates ($\theta$, $\phi$).
We consider a small domain of the sphere, as illustrated in \autoref{fig:sphere-small-domain}.
\begin{figure}[H]
    \centering
    \includegraphics[width=0.5\textwidth]{C:/Users/Matteo/Shallow-Water-Equations/figs/Sphere-small-domain.png}
    \caption{Illustrations of a small domain of the surface of the sphere.}\label{fig:sphere-small-domain}
\end{figure}
We want to find expressions for $\Delta x_{\phi}$ and $\Delta x_{\theta}$, the distances in the $\phi$ and $\theta$ directions, respectively, as illustrated in \autoref{fig:sphere-small-domain}.
We can find these distances by using the arc length formula.
Recall that the circumreference of a full circle is $2\pi r$, where $r$ is the radius of the circle.
The arc length is a fraction of the full circumreference, and it is given by the formula $l = r v$, where $l$ is the arc length, $r$ is the radius, and $v$ is the angle in radians.
Assuming Earth's latitude side is a circle, we can find the distance $\Delta x_{\phi}$ by using the arc length formula, as:
\begin{align*}
    \Delta x_{\phi} = r \Delta \phi,
\end{align*}
where $\Delta \phi$ is the change in the latitude angle and $r$ is the radius of the sphere.
We assume that Earth is a perfect sphere, meaning that the radius is constant.
Considering the longitude dimension $\Delta x_{\theta}$, we need to make some adjustments, as we can see that the circumreference at equator is larger than at the poles.
That is, we need to consider the radius of the circle at the given latitude $\phi$.
To illustrate this, we consider \autoref{fig:sphere_r_lat}.
\begin{figure}[H]
    \centering
    \includegraphics[width=0.4\textwidth]{C:/Users/Matteo/Shallow-Water-Equations/figs/sphere_r_lat.png}
    \caption{Illustration of the radius of the circle at the given latitude $\phi$.}\label{fig:sphere_r_lat}
\end{figure}
In \autoref{fig:sphere_r_lat}, we consider a right triangle with the hypotenuse as the radius of the sphere, and the adjacent side as the radius of the circle at the given latitude $\phi$.
Thus, we can express the radius of the circle at the given latitude $\phi$ as
\begin{align*}
    r_{lat} = r \cos(\phi).  
\end{align*}
Using that, together with the formula for the arc length, we can find the distance $\Delta x_{\theta}$ as:
\begin{align*}
    \Delta x_{\theta} = r \cos(\phi) \Delta \theta.
\end{align*}
The volume of a small domain of the sphere, as shown in \autoref{fig:sphere-small-domain}, is given by 
\begin{align*}
    V &= \Delta x_{\phi} \Delta x_{\theta} h \\
    &= r^2 h \cos(\phi) \Delta \phi \Delta \theta,
\end{align*}
assuming that the height of the domain is $h$, and that the domain is rectangular.
This is a fair assumption for small values of $\Delta x_{\phi}$ and $\Delta x_{\theta}$.
We also assume that $\phi$ is not too close to the poles, as $\cos(\phi)$ will go to zero at the poles.
We are interested in rate of change of the volume with respect to time, and we can find this by taking the time derivative of the volume.
That is, we consider the partial derivative with respect to time of the volume $V$:
\begin{align}\label{eq:sphere-volume-time-derivative}
    \frac{\partial V}{\partial t} &= r^2 \cos(\phi) \Delta \phi \Delta \theta \frac{\partial h}{\partial t},
\end{align}
where we have utilized that $r$ is constant, and that $\cos(\phi), \Delta \phi$ and $\Delta \theta$ are independent of the time $t$.
We use $u_{\theta}$ and $u_{\phi}$ to denote the velocities in the $\theta$ and $\phi$ increasing directions, respectively.
We are interested in the rate at which fluid volume enters the region from the sides.
We can find this rate by considering the flux of fluid volume through the sides of the domain.
That is, we consider how much fluid volume enters the domain from the $\theta$ direction, and how much fluid volume enters the domain from the $\phi$ direction.
We calculate the influx at the $\theta$ line and the outflux at $\theta + \Delta \theta$ line, see \autoref{fig:sphere-small-domain}, to find the net flux.
The influx is the area of the $\theta$ line times the velocity in the $\theta$ direction at the $\theta$ line.
The influx is 
\begin{align*}
     u_\theta(\theta) h(\theta)  r \Delta \phi,
\end{align*}
where $h(\theta)$ is the height of the water at the $\theta$ line, assumed to be constant along the line.
This way we can compute how much the volume changes due to the influx.
For the outflux we do the same just for the $\theta + \Delta \theta$ line, introducing the notation $\theta ' = \theta + \Delta \theta $ meaning the outflux is
\begin{align*}
    u_\theta(\theta + \Delta \theta) h(\theta + \Delta \theta)  r \Delta \phi
    = u_\theta(\theta') h(\theta')  r \Delta \phi
    .
\end{align*}
The net flux in the $\theta$ direction is the difference between the influx and the outflux, and is given by
\begin{align*}
   \left(  u_\theta(\theta) h(\theta) - u_\theta(\theta ') h(\theta ') \right)  r \Delta \phi.
\end{align*}
We can do the same for the $\phi$ direction, also using the notation $\phi ' = \phi + \Delta \phi$.
Using~\eqref{eq:sphere-volume-time-derivative} and the net fluxes for both directions we can write:
\begin{align}\label{eq:sphere-volume-time-derivative-flux}
    r^2 h \cos(\phi) \Delta \phi \Delta \theta
    = \left( u_\theta(\theta) h(\theta) - u_\theta (\theta ')h(\theta ')  \right) r \Delta \phi
    + \left( u_\phi(\phi) h(\phi)\cos (\phi) - u_\phi (\phi ')h(\phi ') \cos(\phi')  \right) r \Delta \theta.
\end{align}
Since we are interested in the rate of change in the water height $h$ with respect to time, we divide~\eqref{eq:sphere-volume-time-derivative-flux} by the area of the element $r^2 \cos(\phi) \Delta \phi \Delta \theta$.
Hence we get
\begin{align}\label{eq:sphere-derivative-h}
    \frac{\partial h}{\partial t} = \frac{u_\theta(\theta) h(\theta) - u_\theta(\theta ')h(\theta ') }{r \cos(\phi) \Delta \theta} 
    + \frac{u_\phi(\phi) h(\phi)\cos (\phi) - u_\phi (\phi ')h(\phi ') \cos(\phi')}{r \cos (\phi) \Delta \phi},
\end{align}
where $\phi \neq \pm \frac{\phi}{2}$.
By collecting terms to the left hand side and changing the order of the numerators, we can rewrite~\eqref{eq:sphere-derivative-h} as 
\begin{align}\label{eq:sphere-derivative-h-collected}
    \frac{\partial h}{\partial t} + \frac{u_\theta(\theta') h(\theta') - u_\theta(\theta)h(\theta) }{r \cos(\phi) \Delta \theta} 
    + \frac{u_\phi(\phi ')h(\phi ') \cos(\phi') - u_\phi(\phi) h(\phi) \cos(\phi) }{r \cos (\phi) \Delta \phi} = 0.
\end{align}
Next step is to investigate the limit values, as $\Delta \theta$ and $\Delta \phi$ goes to zero.
We can find the limit values by using the definition of the derivative.
The derivative of a function $f$ with respect to $x$ is defined as
\begin{align*}
    f'(x) = \lim_{\Delta x \to 0} \frac{f(x + \Delta x) - f(x)}{\Delta x}.
\end{align*}
We can use this definition to find the limit values in~\eqref{eq:sphere-derivative-h-collected}.
\begin{align*}
    \lim_{\Delta \theta \to 0} \frac{ u_\theta(\theta ')h(\theta ') - u_\theta(\theta) h(\theta) }{\Delta \theta} = \frac{\partial}{\partial \theta} (h u_\theta ),
\end{align*}
and 
\begin{align*}
    \lim_{\Delta \phi \to 0} \frac{ u_\phi(\phi ')h(\phi ') \cos(\phi') - u_\phi(\phi) h(\phi) \cos(\phi) }{\Delta \phi} =  \frac{\partial}{\partial \phi} (h u_\phi \cos(\phi)).
\end{align*}
Inserting these results in~\eqref{eq:sphere-derivative-h-collected} yields
\begin{align*}
    h_t + \frac{1}{r \cos (\phi)} \left( {(h u_\theta)}_{\theta} + {(h u_{\phi} \cos(\phi))}_{\phi}  \right) = 0,
\end{align*}
which is the mass conservation equation in spherical coordinates and is the first equation in the shallow water equations in spherical coordinates.
The next step is to derive the momentum equations in spherical coordinates.
In this case, we focus on the horizontal velocity components, specifically the velocity tangential to the surface of the sphere, i.e., the $\theta$ and $\phi$ velocities.
The vertical velocity is neglected, as the key assumption in the shallow water equations is that the vertical component of the acceleration is negligible.
Additionally, when considering Earth, the water layer is thin compared to the radius of the Earth, referred to as a thin-layer approximation.
We need to express the horizontal velocity $u_h$, which is dependent on the variables $\theta, \phi$ and $t$.
Since $\theta$ and $\phi$ are angles, we introduce the unit vectors $\mathbf{e}_\theta$ and $\mathbf{e}_\phi$ on the surface in the $\theta$ and $\phi$ directions, respectively.
The unit vectors are illustrated in~\autoref{fig:sphere-unit-vectors}.
\begin{figure}[H]
    \centering
    \includegraphics[width=0.2\textwidth]{C:/Users/Matteo/Shallow-Water-Equations/figs/sphere-unit-vectors.png}
    \caption{Illustration of the unit vectors $\mathbf{e}_\theta$ and $\mathbf{e}_\phi$.}\label{fig:sphere-unit-vectors}
\end{figure}
We can express the horizontal velocity $u_h$ in terms of the unit vectors as
\begin{align}\label{eq:sphere-horizontal-velocity}
    u_h(\theta, \phi, t) = u_\theta \mathbf{e}_\theta + u_\phi \mathbf{e}_\phi.
\end{align}
We are then interested in the total derivative of the horizontal velocity $u_h$ in~\eqref{eq:sphere-horizontal-velocity} with respect to time.
The total derivative is given by
\begin{align}\label{eq:sphere-total-derivative}
    \frac{\text{d}u_h}{\text{d}t} = \frac{\partial u_h}{\partial t} + \frac{\text{d}\theta}{\text{d}t} \frac{\partial u_h}{\partial \theta} + \frac{\text{d}\phi}{\text{d}t} \frac{\partial u_h}{\partial \phi}.
\end{align}
If we differentiate the longitude angle $\theta$ with respect to time, we get the angular velocity $\omega_\theta$ in the $\theta$ direction, i.e.,
\begin{align*}
    \frac{\text{d}\theta}{\text{d}t} = \omega_\theta,
\end{align*}
meaning that if $\omega_\theta > 0$, the point is moving eastwards, and if $\omega_\theta < 0$, the point is moving westwards.
Similarly, if we differentiate the latitude angle $\phi$ with respect to time, we get the angular velocity $\omega_\phi$ in the $\phi$ direction, i.e.,
\begin{align*}
    \frac{\text{d}\phi}{\text{d}t} = \omega_\phi,
\end{align*}
meaning that if $\omega_\phi > 0$, the point is moving northwards, and if $\omega_\phi < 0$, the point is moving southwards.
By using the arc length formula, we get that 
\begin{align}\label{eq:sphere-arc-length}
    \frac{\text{d}\theta}{\text{d}t} =  \frac{u_\theta}{r \cos(\phi)},
    \quad \frac{\text{d}\phi}{\text{d}t} = \frac{u_\phi}{r}.
\end{align}
We can now insert~\eqref{eq:sphere-arc-length} into~\eqref{eq:sphere-total-derivative} to find the total derivative of the horizontal velocity split into the $\theta$ and $\phi$ directions:
\begin{equation}\label{eq:sphere-total-derivative-split}
    \begin{aligned}
        \frac{\text{d}u_\theta}{\text{d}t} = \frac{\partial u_\theta}{\partial t} + \frac{u_\theta}{r \cos(\phi)} \frac{\partial u_\theta}{\partial \theta} + \frac{u_\phi}{r} \frac{\partial u_\theta}{\partial \phi}, \\
        \frac{\text{d}u_\phi}{\text{d}t} = \frac{\partial u_\phi}{\partial t} + \frac{u_\theta}{r \cos(\phi)} \frac{\partial u_\phi}{\partial \theta} + \frac{u_\phi}{r} \frac{\partial u_\phi}{\partial \phi}.
    \end{aligned}
\end{equation}
We know that the right hand side of~\eqref{eq:sphere-total-derivative-split} are the given physical forces acting on the fluid.
Earlier in this project, we focused on the shallow water equations in Cartesian coordinates, accounting solely for gravitational forces.
However, in spherical coordinates, additional physical forces must be considered. These include the Coriolis force, centripetal acceleration, and the effects of Earth's curvature.
First we consider the gravitational force acting on the fluid, descibed as $-g \Delta h$, where $g$ is the gravitational constant, and $\Delta h$ is the gradient of the height $h$.
We consider the gradient of $h(\theta, \phi)$:
\begin{align}
    \Delta h  &= \frac{1}{r \cos(\phi)} h_{\theta} \mathbf{e}_{\theta} + \frac{1}{r } h_{\phi} \mathbf{e}_{\phi},
\end{align}
meaning that the gravity force acting on the fluid in the $\theta$ and $\phi$ directions are given by:
\begin{align*}
    &\theta-\text{direction:} \quad -\frac{g}{r \cos(\phi)} h_{\theta},\\
    &\phi-\text{direction:} \quad -\frac{g}{r} h_{\phi}.
\end{align*}
Hence, we obtain the two momentum equations in spherical coordinates as
\begin{equation}
    \begin{aligned}
         {(u_\theta)}_t + \frac{u_\theta}{r \cos(\phi)} {(u_\theta)}_\theta + \frac{u_\phi}{r} {(u_\theta)}_\phi = -\frac{g}{r \cos(\phi)} h_{\theta} + \text{other forces}, \\
        {( u_\phi)}_t + \frac{u_\theta}{r \cos(\phi)} {(u_\phi)}_\theta   + \frac{u_\phi}{r} {(u_\phi)}_\phi = -\frac{g}{r} h_{\phi} + \text{other forces}.
    \end{aligned}
\end{equation}
The next force we consider is the Coriolis force, which is a force that acts on moving objects on the surface of the earth~\cite{Coriolis}.
The Coriolis force is given by $f = 2 \Omega \sin(\phi)$, where $\Omega$ is the angular velocity of the earth.
The Coriolis force in the $\theta$ and $\phi$ directions are then given by:
\begin{align*}
    &\theta-\text{direction:} \quad f u_\phi,\\
    &\phi-\text{direction:} \quad -f u_\theta.
\end{align*}
The last thing we need to take into account when working in the spherical domain is the curvature of the earth.
This adds the following terms:
\begin{align*}
    &\theta-\text{direction:} \quad \frac{u_\theta u_\phi}{r} \tan(\phi),\\
    &\phi-\text{direction:} \quad -\frac{u_\theta^2}{r} \tan(\phi).
\end{align*}
Inserting these forces into the momentum equations, we get the shallow water equations in spherical coordinates:
\begin{equation}
    \left.
    \begin{aligned}
        h_t + \frac{1}{r \cos (\phi)} \left( {(h u_\theta)}_{\theta} + {(h u_{\phi} \cos(\phi))}_{\phi}  \right) &= 0, \\
        {(u_{\theta})}_t  + \frac{u_\theta}{r \cos (\phi)} {(u_\theta)}_\theta + \frac{u_\phi}{r} {(u_\theta)}_{\phi}
        - \frac{u_\theta u_\phi }{r} \tan(\phi) + \frac{g}{r \cos (\phi)} h_\theta - f u_\phi &= 0, \\
        {(u_{\phi})}_t  + \frac{u_\theta}{r \cos (\phi)} {(u_\phi)}_\theta + \frac{u_\phi}{r} {(u_\phi)}_{\phi}
        + \frac{u_\theta^2}{r} \tan(\phi) + \frac{g}{r} h_\phi + f u_\theta &= 0,
    \end{aligned}
    \right\}
\end{equation}
where $r$ is the radius, $(\theta, \phi)$ are the longitude and latitude angles, $h$ is the height of the water, $u_\theta$ and $u_\phi$ are the velocities in the $\theta$ and $\phi$ directions, $g$ is the gravitational constant.

Obtain in vector form:
Multiply with $\cos \phi$ and something else. Consider the paper.

Vector form:
\begin{align}
    \cos(\phi) \mathbf{U}_t + \mathbf{F(U)}_\theta + \cos(\phi) \mathbf{G(U)}_\phi = 0,
\end{align}
where 
\begin{align*}
    \mathbf{U} = \begin{bmatrix} h \\ h u_\theta \\h u_\phi \end{bmatrix},
    \mathbf{F(U)} = \begin{bmatrix} h u_\theta \\ h u_\theta^2 + g h^2  \\ h u_\theta u_\phi \end{bmatrix},
    \mathbf{G(U)} = \begin{bmatrix} h u_\phi \\ h u_\theta u_\phi \\ h u_\phi^2 + \ g h^2 \end{bmatrix}.
\end{align*}



\section{Linearized SWE in Spherical Coordinates}
We consider the linearized shallow water equations in cartesian coordinates with one spatial dimension ($x$) and time ($t$).
\begin{equation}
    \left.
    \begin{aligned}
        h_t + H u_x = 0, \\
        u_t + g h_x = 0,
    \end{aligned}
    \right\}
\end{equation}
where $h$ is the height of the water, $u$ is the velocity of the water, $H$ is the average height of the water, and $g$ is the acceleration due to gravity.
We now consider the linearized shallow water equations in spherical coordinates with one spatial dimension, the longitude ($\theta$), and time ($t$).
That is, we assume constant latitude $\phi$. 

In this section, we will derive the linearized shallow water equations in spherical coordinates.
Later in the project, we will use the finite volume method to solve the linearized shallow water equations in spherical coordinates.
We use the linearized SWE for simplicity.
This section use the sources~\cite{BONEV_2018} and~\cite{Eskilsson_2005}.
The linearized shallow water equations in spherical coordinates with one spatial dimension ($\theta$) and time ($t$) are given by
\begin{equation}\label{eq:linearized_swe_spherical}
    \left.
    \begin{aligned}
        h_t + \frac{H}{r \cos(\phi)} u_\theta = 0, \\
        u_t + g h_\theta + fu = 0,
    \end{aligned}
    \right\}
\end{equation}
where $r$ is the radius of the Earth, $\phi$ is the latitude, and $f$ is the Coriolis parameter, given by $f = 2 \Omega \sin(\phi)$, where $\Omega$ is the angular velocity of the Earth.
We still refer to the equations as the mass and momentum equations, respectively.



We can also write the linearized shallow water equations in spherical coordinates in vector form, without the coriolis force, as
\begin{align}\label{eq:linearized_swe_spherical_vector}
    \mathbf{W}_t + \mathbf{A} \mathbf{W}_\theta = 0,
\end{align}
where $\mathbf{W} =
\begin{bmatrix} h \\ u \end{bmatrix}$ and the coefficient matrix $\mathbf{A}$ is constant and given as:
$\mathbf{A} = \begin{bmatrix} 0 & \frac{H}{r \cos(\phi)} \\ g & 0 \end{bmatrix}$.



