\chapter{Data generation}\label{ch:method}
In this chapter, we outline the process of generating the data required for the data-driven methods, which is a crucial part of the project.
We generate all the data ourselves using our numerical solver.
The quality and relevance of the data directly impact the training and performance of the models.
We begin by detailing the data generation process for the 1D SWE using the FVM with fluxes determined from an exact Riemann solver.
Next, we describe the approach for generating data for the 1D LSWE in spherical coordinates.
We also present the methodology for data generation in the context of the 2D SWE.
Finally, we provide a brief introduction of a plan to generate data for training a spherical Fourier neural operator (SFNO) for applications on a planetary scale.

\section{Data generation using the Finite Volume Method}\label{sec:data_generation_fvm}
In this section we clarify the process of generating data using our numerical solver, that is, the FVM.
The FVM is used to solve the 1D SWE, 1D LSWE on a sphere and the 2D SWE.
We specify the necessary information, such as the initial conditions, the domain, and the parameters used in the data generation process.

\subsection*{1D SWE with exact Riemann solver}
In this section, we present how the so-called true solution is found in the code by solving the Riemann problem exactly.
The true solution is found by solving the Riemann problem exact using a high-resolution grid, and distinguishing between the wet-bed or dry-bed case, and also identifying the shock and rarefaction waves.
First we caculate the wave speeds for the left and right states, respectively, as
\begin{align*}
    c_L = \sqrt{g h_L}, \quad c_R = \sqrt{g h_R},
\end{align*}
which are used to determine the critical water height $h_{\text{crit}}$ as
\begin{align*}
    h_{\text{crit}} = (u_R - u_L) - 2(c_L + c_R).
\end{align*}
If either $h_L \leq 0$ or $h_R \leq 0$, we are in a dry-bed case.
If $h_{\text{crit}} \geq 0$, it indicates that the water depth is critical, and we are also in a dry-bed case.
If none of the above conditions are met, we are in a wet-bed case.
Summarized:
\begin{align*}
    \begin{cases}
        \text{Dry-bed case} & \text{if }  h_L \leq 0, \quad  h_R \leq 0 \text{ or } h_{\text{crit}} \geq 0, \\
        \text{Wet-bed case} & \text{otherwise}.
    \end{cases}
\end{align*}
In a dry-bed case, we first identify the location of the dry region, whether it is on the left, right, or in the middle, and then calculate the wave speeds accordingly.
In a wet-bed case, we compute the characteristics $h_*$ and $u_*$ for the star region.
We then identify the shock and rarefaction waves, and calculate the wave speeds for the left and right states, respectively.
The process is illustrated in \autoref{fig:flowchart_true_solution}.
\begin{figure}[H]
    \centering
    \includegraphics[width=0.8\textwidth]{C:/Users/Matteo/Shallow-Water-Equations/figs/flowchart_true_solution.pdf}
    \caption{Flowchart for generating the solution.}\label{fig:flowchart_true_solution}
\end{figure}
When generating the data, we need an initial condition for the water height $h$ and the velocity $u$.
In this study, we use the Gaussian function as the initial condition for the water height $h$.
That is, we define the initial condition for the water height as
\begin{align}\label{eq:1D_swe_ic_gaussian}
    h(x,0) &= a \exp{\left(\frac{-{(x-\mu)}^2}{2\sigma^2}\right)},
\end{align}
where $a$ is the amplitude of the Gaussian, $\mu$ is the center of the distribution, and $\sigma$ is the standard deviation.
We are working on the domain $x \in [0,1]$ m and the parameters are set to $a = 1$, $g = 9.81 m/s^2$ and $\sigma = 0.1$ m.
The value of $\mu$ is varied to generate different initial conditions, we generate data for $\mu = 0.3$ m and $\mu = 0.5$ m.
The initial conditions for the water height can be seen in \autoref{fig:data_generation_initial}.
\begin{figure}[H]
    \centering
    \includegraphics[width=0.5\textwidth]{C:/Users/Matteo/Shallow-Water-Equations/plots/data_generation_initial.png}
    \caption{The generated data has a Gaussian distribution for the initial water height, with $\mu = 0.3$ m and $\mu = 0.5$ m.}\label{fig:data_generation_initial}
\end{figure}
For the initial velocity $u$, we set it to zero, i.e., $u(x,0) = 0$ m/s, meaning that the water is initially at rest.
The solver is validated by comparing the results with known test cases, such as the dam break problem. 
We use a variable time step size $\Delta t$ determined dynamically using the Courant-Friedrichs-Lewy (CFL) number. 
The CFL is dimensionless and is defined as
\begin{align}\label{eq:CFL_number}
    \text{CFL} = s_{\max} \frac{\Delta t}{\Delta x},
\end{align}
where $s_{\max}$ is the maximum wave speed, $\Delta t$ is the time step size, and $\Delta x$ is the grid spacing.
The value $s_{\max}$ is calculated as 
\begin{align*}
    s_{\max} = \max \left( |u_i| + \sqrt{g h_i}  \right),
\end{align*}
where $u_i$ and $h_i$ are the velocity and water height at grid point $i$, respectively, and $g$ is the gravitational acceleration.
In general, the CFL number should be less than or equal to one for stability.
In our case, we use a CFL number of $0.9$.
The data is generated over the time interval $t = 0.0$ s to $t_{\text{end}} = 1.0$ s.

\subsection*{Truncation error}
When generating data, it is essential to be aware of the truncation error.
Truncation erros arises from approximating the solution of the PDEs using a numerical method, in this case the FVM.
It specifically refers to the difference between the exact solution and the numerical approximation.
The critical question is: after a certain number of time steps, how significant is this error?
If the error becomes too large, it raises concerms about the reliability of the generated data.
Excessive truncation error could compromise the accuracy of the model trained on this data.
Therefore, we must carefully evaluate and mitigate these risks to ensure the quality of the data.
To assess the truncation error, we generate a more accurate solution using a finer grid.
This high-resolution solution serves as a reference for evaluating the numerical approximation.
By comparing the high-resolution solution with the numerical solution, we gain insights into the error introduced by the approximation.

We generate a reference solution for solving the 1D SWE using $N = 1000$ grid points and compare it with the solution for $N = 200$ grid points at the final time step, $t = 1.0$ s.
The results are shown in \autoref{fig:1D_SWE_truncation_error}.
\begin{figure}[H]
    \centering
    \includegraphics[width=0.5\textwidth]{C:/Users/Matteo/Shallow-Water-Equations/plots/truncation_error.pdf}
    \caption{Truncation error for the 1D SWE.}\label{fig:1D_SWE_truncation_error}
\end{figure}
The high-resolution solution and the low-resolution solution are plotted in \autoref{fig:1D_SWE_truncation_error}.
We observe that there is a small difference between the high-resolution solution and the low-resolution solution, as the high-resolution solution has a steeper edge, tending to be more discontinuous.
However, overall the two solutions are almost identical, indicating that the truncation error is negligible.
This suggests that the data generated using the FVM is of high quality and can be used for training the data-driven models.

\subsection*{1D LSWE on a sphere}
We also consider the 1D linearized shallow water equations in spherical coordinates, focusing on a circular domain.
The length of the domain corresponds to a full circle, $L = 2 \pi$ radians and is discretized into $N = 500$ points.
The initial condition for the water height $h$ is specified as a Gaussian function wrapped around the circle, expressed as:
\begin{align}\label{eq:1D_swe_spherical_ic}
    h(\theta, 0) &= h_0 + a \exp \left( \frac{-{(\theta-\mu)}^2}{2 \sigma^2} \right),
\end{align}
where $h_0$ is the mean water height in meters, $a$ is the amplitude of the Gaussian, $\mu$ is the mean value, and $\sigma$ is the standard deviation.
The parameters are $a = 1$  and $\mu = \frac{\pi}{4}$ radians.
We generate data for varying values of $\sigma$ to investigate the effect of the standard deviation on the model performance.
The data is generated for $\sigma = \frac{\pi}{8}, \sigma = \frac{\pi}{16}$ and $\sigma = \frac{\pi}{32}$.
The initial velocity $u$ is set to zero, i.e., $u(\theta,0) = 0$ m/s.
The time step size is fixed and set to $\Delta t = 0.0025$ s.
The initial conditions for the three different $\sigma$ values can be seen in \autoref{fig:swe_spherical_1d_initial_conditions_sigmas}.
\begin{figure}[H]
    \centering
    \begin{subfigure}[b]{0.32\textwidth}
        \centering
        \includegraphics[width=\textwidth]{C:/Users/Matteo/Shallow-Water-Equations/plots/SWE-spherical-1d-initial_conditions_sigma1.pdf}
        \caption{$\sigma = \frac{\pi}{8}$.}\label{fig:swe_spherical_1d_sigma1}
    \end{subfigure}
    \begin{subfigure}[b]{0.32\textwidth}
        \centering
        \includegraphics[width=\textwidth]{C:/Users/Matteo/Shallow-Water-Equations/plots/SWE-spherical-1d-initial_conditions_sigma2.pdf}
        \caption{$\sigma = \frac{\pi}{16}$.}\label{fig:swe_spherical_1d_sigma2}
    \end{subfigure}
    \begin{subfigure}[b]{0.32\textwidth}
        \centering
        \includegraphics[width=\textwidth]{C:/Users/Matteo/Shallow-Water-Equations/plots/SWE-spherical-1d-initial_conditions_sigma3.pdf}
        \caption{$\sigma = \frac{\pi}{32}$.}\label{fig:swe_spherical_1d_sigma3}
    \end{subfigure}
    \caption{Initial conditions for the 1D LSWE in spherical coordinates for different \(\sigma\) values.}\label{fig:swe_spherical_1d_initial_conditions_sigmas}
\end{figure}
From \autoref{fig:swe_spherical_1d_initial_conditions_sigmas}, we observe that the standard deviation \(\sigma\) affects the width of the Gaussian function.
The smaller the $\sigma$, the narrower the Gaussian function, meaning the curves are steeper.
This is to test the different models abilities to handle steep gradients.
The data is generated from $t = 0.0$ s to $t_{\text{end}} = 1.0$ s.

\subsection*{2D SWE}
For the 2D SWE, we also use the Gaussian function as initial condition for the water height $h$, but now in two dimensions:
\begin{align}\label{eq:2D_swe_ic_gaussian}
    h(x,y,0) &= h_0 + a \cdot \exp \left( -\frac{{(x-x_c)}^2 + {(y-y_c)}^2}{2\sigma^2} \right), 
\end{align}
where $h_0$ is the initial water height outside of the Gaussian, $a$ is the amplitude of the Gaussian, $(x_c, y_c)$ is the center of the Gaussian in meters, and $\sigma$ is the standard deviation.
The domain is $x,y \in [0,40]$ m and is discretized into $N$ points in each direction.
We use the parameters $h_0 = 1$ m, $a = 2$, $(x_c, y_c) = (20 \text{m}, 20 \text{m})$, and $\sigma = 2$ m.
The initial velocity $u$ is set to zero, i.e., $u(x,y,0) = 0$.
The initial conditions for the water height can be seen in \autoref{fig:2D_gauss_initial_condition}.
\begin{figure}[H]
    \centering
    \includegraphics[width=0.6\textwidth]{C:/Users/Matteo/Shallow-Water-Equations/plots/2D_gauss_initial_condition.pdf}
    \caption{Initial condition for the 2D problem.}\label{fig:2D_gauss_initial_condition_data_gen}
\end{figure}
To generate the data, we use our FVM solver to solve the 2D SWE with the initial conditions just specified.
The solver is validated by comparing its results with known test cases.
We generate two distinct data sets containing solutions of the 2D SWE.
The first data set is created with a variable time step size $\Delta t$, determined dynamically using the CFL condition, where the CFL number is set to $0.9$.
Data is generated from $t = 0.0$ s to $t_{\text{end}} = 5.0$ s.
To investigate the impact of the grid resolution on the model performance, we generate data for different grid resolutions, with values of $N = 64, N = 128$, and $N = 256$.
This approach also allow us to test the models ability to transfer solutions to different grid resolutions, a critical capability for generalization.

The second data set is designed for long-term predictions, where the time step size is fixed.
Since predictions are made beyond the data, the time step size must be known.
For this data set, we use a constant time step size of $\Delta t = 0.025$ s.
This value was determined by analyzing the time step sizes used in the variable step data generation.
To ensure stability, the time step size must be sufficiently small.
By halving the smallest observed time step size, we obtained $\Delta t = 0.025$ s.
The used grid resolution is $N = 64$.
The data is generated from $t = 0.0$ to $t_{\text{end}} = 15.0$ s, providing a robust data set for training models for long-term predictions.

