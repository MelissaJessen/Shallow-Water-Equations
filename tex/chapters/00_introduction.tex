\chapter{Introduction}

This master thesis focuses on the numerical and data-driven solutions for the Shallow Water Equations (SWE), which are fundamental in computational fluid dynamics.
The project aims to solve the SWE using the Finite Volume Method (FVM) in 2D and spherical coordinates and subsequently exploit machine learning techniques, particularly the Fourier Neural Operator (FNO), to improve solution accuracy and efficiency.
By the end of the project, the goal is to have a robust and efficient computational toolkit for solving the SWE, which will also be useful for simulating important geophysical processes like Kelvin and Rossby waves. 

The Shallow Water Equations (SWE) are a set of hyperbolic partial differential equations that describe the motion of a fluid in a shallow layer of water.
In this project, we will derive the SWE in 1D, 2D and in spherical coordinates, to model the flow of water on the surface of the Earth.
We will go through the Finite Volume Method (FVM) in 1D and 2D, which is a numerical method for solving partial differential equations by dividing the domain into small control volumes and integrating the equations over these volumes.
The method is widely used in computational fluid dynamics to model the behavior of fluid flows.
We will implement the FVM and use it to solve the SWE in 1D for two different dam break problems. 


\section{Motivation}

\section{Literature}
When working in this area it inevitably to mention the work of E. F. Toro, who has written several books on the topic of Riemann solvers and the Finite Volume Method, specifically for the shallow water equations.
In this project, we will use the books \textit{Shock-Capturing Methods for Free-Surface Shallow Flows}~\cite{Toro2001-Shock}, \textit{Riemann Solvers and Numerical Methods for Fluid Dynamics}~\cite{Toro2009-Riemann} and the new book from 2024 \textit{Computational Algorithms for Shallow Water Equations}~\cite{Toro2024} as references.

The course \textit{Advanced Numerical Methods for Environmental Models} at the University of Trento, has provided a good foundation for the numerical methods used in this project, both in terms of lecture notes and exercises~\cite{trento_course}.

In the field of Fourier Neural Operators, there is not a lot of literature on the topic, as the concept of Fourier Neural Operators is relatively new.
However, the paper \textit{Fourier Neural Operator for Parametric Partial Differential Equations}~\cite{FNO_2021}, written by several authors, is a key reference.
In the last years the company Nvidia has done some very interesting work on the topic of FNO, and they have published several blog posts on the topic.
One of the posts consider the use Spherical Fourier Neural Operators (SFNO) to generate weather forecasts around the globe\cite{Nvidia2023}.

\section{Thesis overview}




