\section{Metod of manufactured solutions}
Method of manufactured solutions (MMS) was used to verify the implementation.
We use a manufactured solution for $h(x,t)$ and $u(x,t)$.
Choose a simple sine wave function for the water height $h$ and a corresponding $u$ that satisfies the shallow water equations.
Consider
\begin{align*}
    h(x,t) &= h_0 + A \cos(\omega t - kx), \\
    u(x,t) &= \frac{ A \omega }{k h_0}  \cos(\omega t - kx),
\end{align*} 
where $h_0$ is the constant base depth, $A$ is the amplitude of the wave, $k$ is the wave number, and $\omega$ is the angular frequency.

We begin by computing the source terms $S_h$ and $S_u$.
First we compute the partial derivatives
\begin{align*}
    h_t &= ,\\
    u_t &=  \\
\end{align*}
which gives (using the chain rule)
\begin{align*}
    {(hu)}_x &= h_x u + h u_x \\
    &= 
\end{align*}



\begin{figure}[htbp]
    \centering
    % First row
    \begin{subfigure}[b]{0.4\textwidth}
        \centering
        \includegraphics[width=\textwidth]{C:/Users/Matteo/Shallow-Water-Equations/figs/waves-LRRS.png}
        \caption{Left rarefaction wave, right shock wave.}\label{fig:waves-LRRS}
    \end{subfigure}
    \hspace{0.02\textwidth} % Small horizontal space between figures
    \begin{subfigure}[b]{0.4\textwidth}
        \centering
        \includegraphics[width=\textwidth]{C:/Users/Matteo/Shallow-Water-Equations/figs/waves-LSRR.png}
        \caption{Left shock wave, right rarefaction wave.}\label{fig:waves-LSRR}
    \end{subfigure}
    
    % Second row
    \begin{subfigure}[b]{0.4\textwidth}
        \centering
        \includegraphics[width=\textwidth]{C:/Users/Matteo/Shallow-Water-Equations/figs/waves-LRRR.png}
        \caption{Left and right rarefaction waves.}\label{fig:waves-LRRR}
    \end{subfigure}
    \hspace{0.02\textwidth} % Small horizontal space between figures
    \begin{subfigure}[b]{0.4\textwidth}
        \centering
        \includegraphics[width=\textwidth]{C:/Users/Matteo/Shallow-Water-Equations/figs/waves-LSRS.png}
        \caption{Left and right shock waves.}\label{fig:waves-LSRS}
    \end{subfigure}
    \caption{Four possible wave patterns in the solution of the Riemann problem.}\label{fig:wave-patterns}
\end{figure}


\subsection{Flux-splitting/Upwind}
In the flux splitting we compute $h_*$ and $u_*$ by assuming a two-rarefaction wave structure.
We define 
\begin{align*}
    c_L = \sqrt{g h_L}, \quad c_R = \sqrt{g h_R},
\end{align*}
to obtain
\begin{align*}
    h_{S2} = {\left( \frac{0.75}{\sqrt{g}} \cdot (q_L - q_R) + 0.5 \cdot \left(h_L^{1.5} + h_R^{1.5}\right) \right)}^2, \\
    h_{S} = h_{S2}^{\frac{1}{3}}
\end{align*}
and
\begin{align*}
    u_* = \frac{1}{2} (u_L + u_R) + \frac{1}{3} \sqrt{g} (h_L^{1.5} - h_R^{1.5}).
\end{align*}




\subsection{Roe solver}
Consider the non-linear Riemann problem in~\eqref{eq:Riemann_problem}:
\begin{align*}
    \mathbf{U}_t + \mathbf{F(U)}_x \equiv \mathbf{U}_t + \mathbf{A} \mathbf{U}_x = 0,
\end{align*}
where $\mathbf{A}$ is the Jacobian matrix of $\mathbf{F}$. 
The Roe solver is based on an approximation of the Jacobian matrix $\mathbf{A}$ by a Roe matrix $\tilde{\mathbf{A}}$, which is a constant coefficient matrix, to obtain the linear system
\begin{align*}
    \mathbf{U}_t + \mathbf{\tilde{A}} \mathbf{U}_x = 0.
\end{align*}

This means that the Roe solver solves the approximated Riemann problem
\begin{align*}
    \mathbf{U}_t + \mathbf{\tilde{A}} \mathbf{U}_x &= 0. \\
    \mathbb{U}(x,0) = \begin{cases}
        \mathbf{U}_L, & x < 0, \\
        \mathbf{U}_R, & x > 0.
    \end{cases}
\end{align*}
exact.
That is, the original non-linear conservation laws are replaced by a linearised system with constant coefficients.

The main idea in the Roe solver is to find average values $\tilde{h}, \tilde{a}, \tilde{u}$ and $\tilde{\psi}$ for the depth $h$, the celerity $a$ (??), the velocity component $u$ and the scalar $\psi$.
The method thus use the following Roe averages:
\begin{align}\label{eq:Roe_averages}
    \left\{
\begin{aligned}
    \tilde{h} &= \sqrt{h_L h_R}, \\
    \tilde{u} &= \frac{u_L \sqrt{h_L} + u_R \sqrt{h_R}}{\sqrt{h_L} + \sqrt{h_R}}, \\
    \tilde{a} &= \sqrt{\frac{1}{2}(a_L^2 + a_R^2)}, \\
    \tilde{\psi} &= \frac{\psi_L \sqrt{h_L}  + \psi_R \sqrt{h_R}}{\sqrt{h_L} + \sqrt{h_R}}.
\end{aligned}
\right.
\end{align}
The average eigenvalues (of what?) are
\begin{align*}
    \tilde{\lambda}_1 = \tilde{u} - \tilde{a}, \quad \tilde{\lambda}_2 = \tilde{u}, \quad \tilde{\lambda}_3 = \tilde{u} + \tilde{a}, \\
\end{align*} 
with the correspinding right eigenvectors
\begin{align*}
    \tilde{\mathbf{R}}^{(1)} = \begin{bmatrix}
        1 \\ \tilde{u} - \tilde{a} \\ \tilde{\psi}
    \end{bmatrix}, \quad
    \tilde{\mathbf{R}}^{(2)} = \begin{bmatrix}
        0 \\ 0 \\  1
    \end{bmatrix}, \quad
    \tilde{\mathbf{R}}^{(3)} = \begin{bmatrix}
        1 \\ \tilde{u} + \tilde{a} \\ \tilde{\psi}
    \end{bmatrix}.
\end{align*}
The wave strenghts $\tilde{\alpha}_j$ described by Roe averages are given by
\begin{align}
    \left\{
    \begin{aligned}
        \tilde{\alpha}_1 &= \frac{1}{2} \left[ \Delta h - \frac{\tilde{h}}{\tilde{a}} \Delta u  \right], \\
        \tilde{\alpha}_2 &= \frac{1}{2} \left[ \tilde{h} \Delta \psi  \right], \\
        \tilde{\alpha}_3 &= \frac{1}{2} \left[ \Delta h + \frac{\tilde{h}}{\tilde{a}} \Delta u  \right].
    \end{aligned}    
    \right.
\end{align}
Applying theory of linear systems with constant coefficients.
The numerical flux is
\begin{align}\label{eq:Roe_flux}
    \mathbf{F}_{i+\frac{1}{2}} = \frac{1}{2} \left( \mathbf{F}_L + \mathbf{F}_R \right) - \frac{1}{2} \sum_{j=1}^3 \tilde{\alpha}_j \left| \tilde{\lambda}_j \right| \tilde{\mathbf{R}}^{(j)}.
\end{align}
The Roe flux~\eqref{eq:Roe_flux} is used in the explicit conservative scheme to solve the SWE in 1D.

Godunov:
Consider the initial-boundary value problem (IBVP) for a system of $N$ nonlinear hyperbolic conservation (balance?) laws     
\begin{equation}\label{eq:IBVP_system}
    \begin{cases}
    \text{PDEs: }    &\mathbf{U}_t + \mathbf{F(U)}_x = \mathbf{S(U)}, \quad x \in [a, b], \quad t > 0, \\
    \text{ICs: }    &\mathbf{U}(x,0) = \mathbf{U}^{(0)}(x), \quad x \in [a,b], \\
    \text{BCs: }    &\mathbf{U}(a,t) = \mathbf{B}_{L}(t), \quad \mathbf{U}(b,t) = \mathbf{B}_{R}(t), \quad t \geq 0.
    \end{cases}
\end{equation}
The vectors $\mathbf{B}_L (t)$ and $\mathbf{B}_R (t)$ denote the boundary conditions at the left and right boundaries, respectively.
The Godunov Upwind method in conservative form~\eqref{eq:explicit_conservative_1D_SWE} solves the IBVP~\eqref{eq:IBVP_system}.
We compute $h_*$ and $u_*$ by starting with a two-rarefaction wave structure, and solve the wetbed problem iteratively.


Derive FVM scheme 1D SWE spherical:




\begin{align}
    \int_{\theta_L}^{\theta_R} h_t + \frac{H}{r \cos(\phi)} u_\theta \text{ d}\theta = 0 \\
    \int_{\theta_L}^{\theta_R} h_t \text{ d}\theta + \int_{\theta_L}^{\theta_R} \frac{H}{r \cos(\phi)} u_\theta \text{ d}\theta = 0.
\end{align}
Then we integrate the momentum equation in~\eqref{eq:linearized_swe_spherical} over $\theta$ from $\theta_L$ to $\theta_R$ to obtain
\begin{align*}
    \int_{\theta_L}^{\theta_R} u_t \text{ d}\theta + \int_{\theta_L}^{\theta_R} g h_\theta \text{ d}\theta + \int_{\theta_L}^{\theta_R} fu \text{ d} \theta = 0.
\end{align*}


The first term is 
\begin{align*}
    \frac{\partial}{\partial t} \int_{\theta_L}^{\theta_R} h \text{ d}\theta =  \Delta \theta h_t,
\end{align*}
meaning that the first term is the rate of change of the water height $h$ in the control volume.
The second term is
\begin{align*}
    \int_{\theta_L}^{\theta_R} \frac{H}{r \cos(\phi)} u_\theta \text{ d}\theta = \frac{H}{r \cos(\phi)} (u_R - u_L),
\end{align*}
where $u_R$ and $u_L$ are the velocities at the right and left boundaries of the control volume, respectively.



The first term gives the time rate of change in the velocity $u$ in the control volume:
\begin{align*}
    \frac{\partial}{\partial t} \int_{\theta_L}^{\theta_R} u \text{ d}\theta = \Delta \theta u_t.
\end{align*}
The second term gives 
\begin{align*}
    \int_{\theta_L}^{\theta_R} g h_\theta \text{ d}\theta = g(h_R - h_L),
\end{align*}
where $h_R$ and $h_L$ are the heights at the right and left boundaries of the control volume, respectively.
The third term with the Coriolis force gives
\begin{align*}
    \int_{\theta_L}^{\theta_R} fu \text{ d}\theta = f(u_R - u_L) \Delta \theta.
\end{align*} 
Hence, the integral form of the linearized shallow water equations in spherical coordinates with one spatial dimension and time is

Then we integrate~\eqref{eq:integral_form_spherical_1D} over time from $t_1:= t_n$ to $t_2:= t_{n+1}$:
\begin{equation}
    \begin{aligned}
        \int_{t_1}^{t_2} \int_{\theta_L}^{\theta_R} h_t \text{ d}\theta \text{d}t + \int_{t_1}^{t_2} \frac{H}{r \cos(\phi)} (u_R - u_L) \text{ d}t &= 0, \\
        \int_{t_1}^{t_2} \int_{\theta_L}^{\theta_R} u_t \text{ d}\theta \text{d}t + \int_{t_1}^{t_2} g(h_R - h_L) \text{ d}t + \int_{t_1}^{t_2} f(u_R - u_L) \Delta \theta &= 0.
    \end{aligned}
\end{equation}
which equals
\begin{equation}\label{eq:integral_form_spherical_1D_2}
    \begin{aligned}
        \int_{\theta_L}^{\theta_R} h(\theta, t_2) \text{ d}\theta &= \int_{\theta_L}^{\theta_R} h(\theta, t_1) \text{ d}\theta - \frac{H}{r \cos(\phi)} (u_R - u_L) \Delta t, \\
        \int_{\theta_L}^{\theta_R} u(\theta, t_2) \text{ d}\theta &= \int_{\theta_L}^{\theta_R} u(\theta, t_1) \text{ d}\theta - g(h_R - h_L) \Delta t - f(u_R - u_L) \Delta \theta \Delta t.
    \end{aligned}
\end{equation}
We divide~\eqref{eq:derive_integral_form_1D_2} with the cell length $\Delta \theta$ to obtain 
\begin{equation}\label{eq:derive_integral_form_spherical}
    \begin{aligned}
        \frac{1}{\Delta \theta} \int_{\theta_L}^{\theta_R} h(\theta, t_2) \text{ d}\theta &= \frac{1}{\Delta \theta} \int_{\theta_L}^{\theta_R} h(\theta, t_1) \text{ d}\theta -  \frac{\Delta t}{\Delta \theta} \frac{H}{r \cos(\phi)} (u_R - u_L), \\
        \frac{1}{\Delta \theta} \int_{\theta_L}^{\theta_R} u(\theta, t_2) \text{ d}\theta &= \frac{1}{\Delta \theta} \int_{\theta_L}^{\theta_R} u(\theta, t_1) \text{ d}\theta - \frac{\Delta t}{\Delta \theta} g(h_R - h_L)  - f(u_R - u_L) \Delta t.
    \end{aligned}
\end{equation}
By inserting the cell averages in~\eqref{eq:derive_integral_form_spherical}, we obtain the finite volume scheme for the linearized shallow water equations in spherical coordinates with one spatial dimension and time:
\begin{equation}
    \left.
    \begin{aligned}
        h_i^{n+1} &= h_i^n - \frac{\Delta t}{\Delta \theta} (F_{h, i + \frac{1}{2}} - F_{h, i - \frac{1}{2}}),  \\
        u_i^{n+1} &=  u_i^n - \frac{\Delta t}{\Delta \theta} (F_{u, i + \frac{1}{2}} - F_{u, i - \frac{1}{2}}) - \Delta t f(u_{i}^n).
    \end{aligned}
    \right\}
\end{equation}
Where:
\begin{equation}
    \begin{aligned}
        h_i^n &= \frac{1}{\Delta \theta} \int_{\theta_{i-1/2}}^{\theta_{i+1/2}} h(\theta, t_n) \text{ d}\theta, \\
        u_i^n &= \frac{1}{\Delta \theta} \int_{\theta_{i-1/2}}^{\theta_{i+1/2}} u(\theta, t_n) \text{ d}\theta, \\
        F_{h, i + \frac{1}{2}} &= \frac{H}{r \cos(\phi)} (u_{i+1} - u_i), \\
        F_{h, i - \frac{1}{2}} &= \frac{H}{r \cos(\phi)} (u_{i} - u_{i-1}), \\
        F_{u, i + \frac{1}{2}} &= g(h_{i+1} - h_i).
    \end{aligned}
\end{equation}


Now that we have the integral form, we will obtain the finite volume scheme for the linearized shallow water equations in spherical coordinates, by dividing the integral form~\eqref{eq:integral_form_spherical_1D} by the cell length $\Delta \theta$:
\begin{equation}
    \left.
    \begin{aligned}
        h_t + \frac{H}{r \cos(\phi)} \frac{(u_R - u_L)}{\Delta \theta} &= 0, \\
        u_t +  \frac{g (h_R - h_L)}{\Delta \theta} + f(u_R - u_L) &= 0.
    \end{aligned}
    \right\}
\end{equation}


Putting it all together we thus have:
\begin{equation}
    \begin{aligned}
        \Delta \theta h_t + \frac{H}{r \cos(\phi)} (u_R - u_L) &= 0, \\
        \Delta \theta u_t + g(h_R - h_L) + f(u_R - u_L) \Delta \theta &= 0.
    \end{aligned}
\end{equation}
Rewritten as the discretized equations:
\begin{equation}
    \left.
    \begin{aligned}
        h_t &= - \frac{H}{r \cos(\phi)} \frac{(u_R - u_L)}{\Delta \theta}, \\
        u_t &= - \frac{g (h_R - h_L)}{\Delta \theta}  - f(u_R - u_L).
    \end{aligned}
    \right\}
\end{equation}    
Then we calculate the fluxes at the boundaries of the control volume.
In this case we use the most simple fluxes, the average of the velocities at the boundaries, i.e., for cell $i$:
\begin{equation}
    \begin{aligned}
        h_R &= \frac{1}{2} (h_{i+1} + h_i), \\
        h_L &= \frac{1}{2} (h_i + h_{i-1}), \\
        u_R &= \frac{1}{2} (u_{i+1} + u_i), \\
        u_L &= \frac{1}{2} (u_i + u_{i-1}).
    \end{aligned}
\end{equation}
Finally we integrate with respect to time to obtain the finite volume scheme for the linearized shallow water equations in spherical coordinates:
When discretizing the equations, we define the cell averages $h_i^n$ and $u_i^n$ as the average values of the height and velocity in the $i$-th cell at time $t_n$:

Numerical flux\ldots

Time integration.



