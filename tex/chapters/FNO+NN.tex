\section{Fourier Neural Operators and Neural Networks}
The FVM, together with other numerical solvers such as the FDM and FEM (Finite Element Method), solves PDEs by discretizing the domain into a grid.
The finer the grid, the more accurate the solution, but also the more computationally expensive the solution. This introduces a trade-off between accuracy and computational cost.
Complex PDEs often require a fine grid to capture the solution accurately, which can be computationally expensive.
The hope for data-driven methods is that, by learning the dynamics of the solution, we can reduce the computational cost of solving PDEs and still maintain a high level of accuracy.
A classical neural network (NN), is able to learn a map from input to output, i.e., a map between finite-dimensional spaces.
Fourier Neural Operators stands out at they are able to learn mappings between function spaces, meaning they are also grid-independent.



\subsection{Fourier Neural Operators}
In this section, we will introduce the concept of Fourier Neural Operators (FNO).
The theory and method described here is based on the paper~\cite{FNO_2021}.

We consider the operator $G: A \to U$, that maps from a infinite-dimensional function space $A$ to another infinite-dimensional function space $U$.
We aim to approximate the exact operator $G$ by constructing the map
\begin{align*}
    G_{\theta}: A \mapsto, \quad \theta 
\end{align*} 

