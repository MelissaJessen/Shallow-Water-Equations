\section{Fourier Neural Operators and Neural Networks}
The FVM, together with other numerical solvers such as the FDM and FEM (Finite Element Method), solves PDEs by discretizing the domain into a grid.
The finer the grid, the more accurate the solution, but also the more computationally expensive the solution. This introduces a trade-off between accuracy and computational cost.
Complex PDEs often require a fine grid to capture the solution accurately, which can be computationally expensive.
The hope for data-driven methods is that, by learning the dynamics of the solution, we can reduce the computational cost of solving PDEs and still maintain a high level of accuracy.
A classical neural network (NN), is able to learn a map from input to output, i.e., a map between finite-dimensional spaces.
Fourier Neural Operators stands out at they are able to learn mappings between function spaces, meaning they are also grid-independent.
We utilize the fact that differentation is equivalent to multiplication in the Fourier domain.


\subsection{Fourier Neural Operators}
In this section, we will introduce the concept of Fourier Neural Operators (FNO).
The theory and method described here is based on the paper~\cite{FNO_2021}.

We consider the operator $G: A \to U$, that maps from a infinite-dimensional function space $A$ to another infinite-dimensional function space $U$.
We aim to approximate the exact operator $G$ by constructing the map
\begin{align*}
    G_{\theta}: A \mapsto U, \quad \theta \in \Theta,
\end{align*} 
where $\Theta$ is a finite-dimensional parameter space.
Consider the functions $a \in A$ and $u \in U$.
We can access the data by point wise evaluations of the functions, i.e., we have access to the observations ${\{a_j, u_j \}}_{j=1}^N$, in a domain $D \subset \mathbb{R}^d$, which is bounded open set.
The neural operator is iterative, where the update $v_t \mapsto v_{t+1}$ is defined as
\begin{align}
    v_{t+1}(x) := \sigma \left( W v_t(x) + \left( \mathcal{K}(a;\phi)v_t \right) (x) \right), \quad \forall x \in D
\end{align}
where $W: \mathbb{R}^{d_v} \to \mathbb{R}^{d_v}$ is a linear transformation, and $\sigma: \mathbb{R} \to \mathbb{R}$ is a non-linear activation function.
We define the Fourier integral operator $\mathcal{K}$ as 
\begin{align}
    \left( \mathcal{K}(\phi)v_t \right) (x) = \mathcal{F}^{-1} \left( R_{\phi} \cdot (\mathcal{F}v_t ) \right)(x), \quad \forall x \in D
\end{align}
where $R_{\phi}$ is the Fourier transform of a periodic function. 
We aim for a multi-step prediction model, which can predict a given number of time steps.


