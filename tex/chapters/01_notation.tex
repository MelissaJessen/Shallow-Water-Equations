\section{Notation}\label{sec:notation}
Before deriving the shallow water equations (SWE), we will introduce the notation that will be used throughout this report.
In both the 1D case and the 2D case of the SWE, we use cartesian coordinates $(x, y, z)$ with time denoted by $t$.
Given that linear algebra is a fundamental tool used in this report, we first establish the relevant notation.
Lowercase bold letters represent vectors, while uppercase bold letters represent matrices.
For instance, $\mathbf{a}$ is a vector of size $r \times 1$, where $r \in \mathbb{R}$, and $\mathbf{A}$ is a matrix of size $m \times n$ with $m,n \in \mathbb{R}$.
The identity matrix, denoted by $\mathbf{I}$, is a square matrix with ones along the diagonal and zeros elsewhere.
For example, the $3 \times 3$ identity matrix is given by:
\begin{align*}
    \mathbf{I} = \begin{bmatrix}
        1 & 0 & 0 \\
        0 & 1 & 0 \\
        0 & 0 & 1
    \end{bmatrix}.
\end{align*}
In this project, differential calculus plays a significant role.
We denote partial derivatives using the following notation:
\begin{align}\label{eq:notation_partial_derivatives}
   f_x =  \frac{\partial f}{\partial x}, \quad f_y = \frac{\partial f}{\partial y}, \quad f_z = \frac{\partial f}{\partial z}.
\end{align}
The gradient operator, denoted by $\nabla$, gives the gradient of a scalar function $f(x,y,z)$ as a vector:
\begin{align*}
    \nabla f = \begin{bmatrix}
        \frac{\partial f}{\partial x} &
        \frac{\partial f}{\partial y} &
        \frac{\partial f}{\partial z}
    \end{bmatrix}.
\end{align*}
Given two vectors $\mathbf{a} = \begin{bmatrix}
    a_1 & a_2 & a_3
\end{bmatrix}^\top $ and $\mathbf{b} = \begin{bmatrix}
    b_1 & b_2 & b_3
\end{bmatrix}^\top$, the dot product of $\mathbf{a}$ and $\mathbf{b}$ is given by:
\begin{align*}
    \mathbf{a} \cdot \mathbf{b} = a_1 b_1 + a_2 b_2 + a_3 b_3.
\end{align*}
The dot product can also be written as a matrix product:
\begin{align*}
    \mathbf{a} \cdot \mathbf{b} = \mathbf{a}^\top \mathbf{b}.
\end{align*}
The divergence operator, represented as $\nabla \cdot $, gives the divergence of a vector $\mathbf{a}$ as:
\begin{align*}
    \nabla \cdot \mathbf{a} = \frac{\partial a_1}{\partial x} + \frac{\partial a_2}{\partial y} + \frac{\partial a_3}{\partial z} = {a_1}_x + {a_2}_y + {a_3}_z,
\end{align*}
using the notation for partial derivatives introduced in~\eqref{eq:notation_partial_derivatives}.
The tensor product of two vectors $\mathbf{a}$ and $\mathbf{b}$, denoted by $\mathbf{a} \otimes \mathbf{b}$, is a matrix where each element is the product of the elements of $\mathbf{a}$ and $\mathbf{b}$, i.e.,
\begin{align*}
    \mathbf{a} \otimes \mathbf{b} = \begin{bmatrix}
        a_1 b_1 & a_1 b_2 & a_1 b_3 \\
        a_2 b_1 & a_2 b_2 & a_2 b_3 \\
        a_3 b_1 & a_3 b_2 & a_3 b_3
\end{bmatrix}.
\end{align*}
Establishing this relevant notation, we can now derive the shallow water equations.



