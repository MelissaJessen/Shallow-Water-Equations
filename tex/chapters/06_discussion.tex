\chapter{Discussion}\label{ch:discussion}
This chapter discusses the results presented and the methods used in this thesis, focusing on their advantages and limitations.
The discussion is structured around key aspects of the FVM, data-driven approaches and their potential applications in solving the SWE.

Based on the test cases in \autoref{ch:numerical_results}, the FVM implemented in this thesis has proven to be a reliable method for solving the SWE.
When solving PDEs, numerical methods in general have the advantage of being able to control errors.
By refining the grid size or time step, it is possible to estimate and reduce errors, ensuring high accuracy in the solution.
This capability is crucial for applications requiring precise simulations.
For some cases of modelling flood dynamics or tsunamis, where the accuracy of the simulation is crucial, the FVM is a suitable method.
However, the reliance on fine grid resolutions for higher accuracy comes at the cost of increased computational time.
This limitation makes numerical methods, and in this case the FVM, less suitable for scenarios demanding rapid simulations, such as real-time flood forecasting, neccesary for decision-making in emergency situations.

The data-driven methods investigated in this thesis, namely the CNN and the FNO, demonstrated in \autoref{ch:data-driven-results} their ability to solve the SWE.
While the accuracy of these models was lower compared to the FVM, this can be attributed to the small training data set and the limited number of epochs used in training.
Training on a larger data set and for more epochs would likely improve the performance of the data-driven methods, but at the cost of additional computational time during training.
A limitation of the data-driven methods is that we do not have the same control over errors as with numerical methods.
It is difficult to estimate errors for data-driven methods, as they do not have the same transparency as numerical methods.
For some applications, this also makes it difficult to let the data-driven methods stand alone.
This lack of transparency makes it challenging to assess the reliability of the data-driven methods and the accuracy of their predictions.
A good understanding of the data-driven methods is essential to determine when they can be used in real-world applications.
In this thesis, the models were trained by minimizing the loss function for the validation data, with the test data reserved for the final evaluation of the models.
By computing the loss on test data, we gain insight into the accuracy of the models.
However, this approach relies on the availability of data for evaluation.
For applications like forecasting future events, where data is naturally unavailable at prediction time, evaluating model performance or estimating errors becomes significantly more challenging.
%This highlights a fundamental limitation of data-driven methods, particularly for tasks requiring extrapolation into unknown scenarios.
%It is not straightforward to estimate the error of the data-driven methods, which is a disadvantage when compared to numerical methods.
This lack of transparency is a significant drawback, as it makes it difficult to assess the reliability of the data-driven methods and the accuracy of their predictions.
One way to address this issue is to use Physics-Informed Neural Networks (PINNs), which incorporate the physics of the problem into the neural network, making the model more transparent and interpretable.

A significant advantage of the data-driven methods is their ability to generalize to unseen data.
This was demonstrated in the 1D SWE case, where both the CNN and FNO successfully predicted solutions for new initial conditions in less than one second.
The FNO maintained high accuracy, while the CNN experienced a slight drop in precision.
This capability makes data-driven methods particularly valuable for applications requiring rapid predictions.
Another area where the data-driven methods excel is in transferring solutions across grids.
Both the CNN and FNO models maintained accuracy when transtioning solutions from a coarse grid to a fine grid, with the FNO outperforming the CNN in terms of MSE and MAE.
This ability to transfer solutions across grids is a promising feature of data-driven methods, as it can help address scalability issues associated with numerical methods.
By training and saving a model for use across various grid sizes, data-driven methods offer a practical alternative to numerical methods when speed is a priority.
Additionally, their shorter computation time is a sustainability aspect, as it reduces the energy consumption of the computations.
From the literature, it is also suggested that grid transferability is one of the most notable strengts of FNOs.

The literature highlights the strong potential of FNOs for long-term predictions, and this study demonstrated promising results in this area.
By training an FNO model on just 8 seconds of data, the model was able to predict solutions up to 5 seconds into the future while maintaining accuracy.
This capability is particularly valuable for applications requiring long-term predictions, such as flood forecasting or climate modelling.
The CNN model also achieved acceptable results for long-term predictions, but its accuracy decreased at longer time horizons.
This increasing error highlights a limitation of the CNN model for long-time forecasts compared to the FNO model.

Once trained and saved, both models can generate predictions rapidly, offering practical alternatives to numerical methods when speed is a priority.
Nevertheless, numerical methods retain the advantage of higher accuracy, particularly for applications requiring precise simulations.

Another crucial aspect to consider is the quality of the data used to train the models.
Data-driven methods are only as good as the data they are trained on, and their performance depends on the quality and variety of the data.
When training data-driven models on FVM-generated data, we risk accumulating errors that could affect the accuracy of the predictions.
To address this issue, we tested the truncation error in \autoref{sec:data_generation_fvm} and found that the error was very small.
The data generated from the FVM is clean and accurate, but it may not fully represent the complexity of real-world scenarios.
While it solves the SWE effectively, real-world applications may involve additional factors that influence water height and velocity, such as wind direction and velocity, which the SWE do not account for.
This highlights a limitation of the SWE when applied to real-world scenarios.
A notable advantage of data-driven methods is that they do not require the PDE to solve the system, only the data itself.
This makes them useful for applications where we do not have a PDE to solve the system or where the PDE is unknown.
However, the lack of transparency and interpretability of data-driven methods can be a drawback, as it makes it challenging to understand how the models arrive at their predictions.

When dealing with real-world data, additional challenges arise, including the presence of noise, which can impact the accuracy of the models.
High-quality and enough data are essential for training these methods to perform effectively.
This emphasizes the importance of adressing these challenges when applying data-driven methods to real-world problems.



%\begin{itemize}
 %   \item The data-driven methods could possibly perform better if trained on more data and for more epochs.  
  %  \item Has somehow shown potential for what the data-driven methods can do? Supplement to numerical methods, depending on the application.
   % \item The data-driven methods are not as transparent as the numerical methods. Or accurate.
    %\item Truncation errors. Generating data fron the FVM, how much accuracy do we lose?
%\end{itemize}




