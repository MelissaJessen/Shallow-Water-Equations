\chapter{Discussion}\label{ch:discussion}
This chapter discuss the results presented and the methods used in this thesis.

Regarding the FVM implemented in this thesis and used to solve the test cases in \autoref{ch:numerical_results}, it looks correct as it is able to solve the test cases.
In general the FVM is a recognized method for solving PDEs.
In general for numerical methods to solve PDEs we are able to control errors. 
Meaning it is possible to estimate the error and control it by changing the grid size or the time step.
This is an advantage for the numerical methods, and is important, depending on the application, but especially if we consider an application where we need a high accuracy, and to also know the accuracy of the solution.
This is more challenging when working with data-driven methods, as we do not have the same control over the error, and it is not straight forward to estimate the error.
This is a disadvantage for the data-driven methods, and is important to consider when choosing a method to solve a PDE.
This also shows another issue with the data-driven methods, as they are not as transparent as the numerical methods.
One way to tackle this issue is to consider the use of Physics Informed Neural Networks (PINNs), where we can incorporate the physics into the neural network, and thereby make the neural network more transparent.


In \autoref{ch:data-driven-results}, we have demonstrated that the data-driven methods are able to solve the test cases.
As they are trained on data generated by the FVM, they do not obtain the same accuracy. 
It is also worth noting that the data-driven methods are trained on a very small data set, for only one case.
Probably, if the data-driven methods were trained on more data, and for more epochs, they would perform better.
However, that would also lead to an increased computational time. 
Overall, the methods are able to solve the test cases, but where thet really excel is in their ability to generalize to unseen data. 
We see this in the 1D SWE case where both the CNN and FNO are able to solve the problem, for a new initial condition in less than one second.
The FNO was able to maintain the accuracy, where the accuracy for the CNN model dropped a litle.
This a big advantage for the data-driven methods compared to the numerical methods. 
This means that it is possible to have trained and saved a CNN or FNO model, and then make predictions on new data in a very short time. 

Another area where the data-driven methods excel is their ability to transfer solutions across grids. 
For both the CNN and FNO models we demonstrated good results when transferring solutions across grids.
The accuracy compared to the coarse grid was maintained.
This shows great potential for the use of data-driven methods in handling some of the scalability issues that comes with numerical methods.
Especially the FNO model showed great potential in this area, as it has a lower MSE and MAE for the transferred solution compared to the CNN model.
Might serve as an alternative to the numerical methods, depending on the application.
Again, it is possible to have trained and saved a CNN or FNO model, and then make predictions on new data, that could be a finer grid, in a very short time.
For some applications, it might be useful.

As long as the data-driven methods are transparent, it is difficult to let them stay alone, as we cannot estimate the errors.
Find a tool that can estimae the errors for data-driven methdos.
In this thesis we train the models after minimizing the loss function for the validation data.
We keep the test data for the final evaluation of the models.
We compute the loss for the test data and get an idea of the accuracy of the models.
This way, require that we have data to evaluate the models on.
For instance making predictions into the future, naturally we do not have the data yet, and thus, it can be difficult to evaluate the models/estimate the errors.

The litterature suggest good results for FNOs in terms of long-time predictions.
In this study we demonstrated that there is definitely potentiaal in this area.
Training a FNO model it was able to make predictions for up to 5 seconds into the future (trained on only 8 seconds), while maintianing accuracy.
Again, we see that for the model once trained and saved it can make these predictions.
Offering an alternative to numerical methods, if we need a fast simulation.
Where numerical mehtods can provide an accurate simulation.

A whole another aspect is when we consider real-world applications, how accurate the SWE are in describing the real-world phenomena.
The SWE may be good to describe the water height and veolocity, however there are also phenomena they do not take into for simuating water behaviour in the real world, such as wind direction and velocity.
When using data-driven like the ones in this thesis, they do not require the PDE but only data itself.
This can also be an advantage for applications where we do not have a PDE to solve the system or the PDE is unknown.
We compromise the transparency of the methods and maybe also the explainability of the methods.

It is also important to be aware of that the performace of the data-driven methods are dependent on the data they are trained on.
Real-world data would contain noise, which is an extra challenge to overcome.
Also for the real-world data most probably there are other factors, such as wind, we must take into account. 
Requires high quality data to train on. 
The data generated from the FVM is clean, but we risk some accumulating errors, if we approximate the real world using the FVM and then train the data-driven methods on this data.
However, to overcome this isssue, we tested the truncation error in \autoref{sec:data_generation_fvm}, and found that the error is small.







\begin{itemize}
    \item How good are the data-driven methods compared to the classic numerical methods, in this case the FVM?
    \item Shorter computation time = sustainability
    \item Advantages/disadvantages numerical / data-driven methods - compare with the literature 
    For the numerical methods we can often estimate the error. IS that possible using data-driven methods?
    \item The data-driven methods could possibly perform better if trained on more data and for more epochs.  
    \item Has somehow shown potential for what the data-driven methods can do? Supplement to numerical methods, depending on the application.
    \item The data-driven methods are not as transparent as the numerical methods. Or accurate.
    \item Truncation errors. Generating data fron the FVM, how much accuracy do we lose?
    \item Perspectivations to the literature.
\end{itemize}






