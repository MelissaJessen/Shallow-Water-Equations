\chapter{Methodology}
In this chapter..

\section{Approximate Riemann solvers}
We will study two approximate Riemann solvers to solve the Riemann problem~\eqref{eq:Riemann_problem}, that is, the HLL and Roe. 

\subsection{HLL solver}
The HLL (Harten, Lax and van Leer) approach assumes a two-wave structure of the Riemann problem, see Figure.
The solver is based on the data $\mathbf{U}_L \equiv \mathbf{U}_i^n, \mathbf{U}_R \equiv \mathbf{U}_{i+1}^n$ and fluxes $\mathbf{F}_L \equiv \mathbf{F}(\mathbf{U}_L), \mathbf{F}_R \equiv \mathbf{F}(\mathbf{U}_R)$.

The HLL flux is given by
\begin{align*}
    \mathbf{F}_{1 + \frac{1}{2}} = \begin{cases}
        \mathbf{F}_L & \text{if } S_L \geq 0, \\
        \mathbf{F}^{hll} \equiv \frac{S_R \mathbf{F}_L - S_L \mathbf{F}_R + S_L S_R (\mathbf{U}_R - \mathbf{U}_L)}{S_R - S_L} & \text{if } S_L \leq 0 \leq S_R, \\
        \mathbf{F}_R & \text{if } S_R \leq 0.
    \end{cases}
\end{align*}
The wave speeds $S_L$ and $S_R$ must be estimated in some way, and one possibility is to use 
\begin{align*}
    S_L = u_L - a_L q_L, \quad S_R = u_R + a_R q_R.
\end{align*}

\subsection{Roe solver}
Consider the non-linear Riemann problem in~\eqref{eq:Riemann_problem}:
\begin{align*}
    \mathbf{U}_t + \mathbf{F(U)}_x \equiv \mathbf{U}_t + \mathbf{A} \mathbf{U}_x = 0,
\end{align*}
where $\mathbf{A}$ is the Jacobian matrix of $\mathbf{F}$. 
The Roe solver is based on an approximation of the Jacobian matrix $\mathbf{A}$ by a Roe matrix $\tilde{\mathbf{A}}$, which is a constant coefficient matrix, to obtain the linear system
\begin{align*}
    \mathbf{U}_t + \mathbf{\tilde{A}} \mathbf{U}_x = 0.
\end{align*}

This means that the Roe solver solves the approximated Riemann problem
\begin{align*}
    \mathbf{U}_t + \mathbf{\tilde{A}} \mathbf{U}_x &= 0. \\
    \mathbb{U}(x,0) = \begin{cases}
        \mathbf{U}_L, & x < 0, \\
        \mathbf{U}_R, & x > 0.
    \end{cases}
\end{align*}
exact.
That is, the original non-linear conservation laws are replaced by a linearised system with constant coefficients.

The main idea in the Roe solver is to find average values $\tilde{h}, \tilde{a}, \tilde{u}$ and $\tilde{\psi}$ for the depth $h$, the celerity $a$ (??), the velocity component $u$ and the scalar $\psi$.
The method thus use the following Roe averages:
\begin{align}\label{eq:Roe_averages}
    \left\{
\begin{aligned}
    \tilde{h} &= \sqrt{h_L h_R}, \\
    \tilde{u} &= \frac{u_L \sqrt{h_L} + u_R \sqrt{h_R}}{\sqrt{h_L} + \sqrt{h_R}}, \\
    \tilde{a} &= \sqrt{\frac{1}{2}(a_L^2 + a_R^2)}, \\
    \tilde{\psi} &= \frac{\psi_L \sqrt{h_L}  + \psi_R \sqrt{h_R}}{\sqrt{h_L} + \sqrt{h_R}}.
\end{aligned}
\right.
\end{align}
The average eigenvalues (of what?) are
\begin{align*}
    \tilde{\lambda}_1 = \tilde{u} - \tilde{a}, \quad \tilde{\lambda}_2 = \tilde{u}, \quad \tilde{\lambda}_3 = \tilde{u} + \tilde{a}, \\
\end{align*} 
with the correspinding right eigenvectors
\begin{align*}
    \tilde{\mathbf{R}}^{(1)} = \begin{bmatrix}
        1 \\ \tilde{u} - \tilde{a} \\ \tilde{\psi}
    \end{bmatrix}, \quad
    \tilde{\mathbf{R}}^{(2)} = \begin{bmatrix}
        0 \\ 0 \\  1
    \end{bmatrix}, \quad
    \tilde{\mathbf{R}}^{(3)} = \begin{bmatrix}
        1 \\ \tilde{u} + \tilde{a} \\ \tilde{\psi}
    \end{bmatrix}.
\end{align*}
The wave strenghts $\tilde{\alpha}_j$ described by Roe averages are given by
\begin{align}
    \left\{
    \begin{aligned}
        \tilde{\alpha}_1 &= \frac{1}{2} \left[ \Delta h - \frac{\tilde{h}}{\tilde{a}} \Delta u  \right], \\
        \tilde{\alpha}_2 &= \frac{1}{2} \left[ \tilde{h} \Delta \psi  \right], \\
        \tilde{\alpha}_3 &= \frac{1}{2} \left[ \Delta h + \frac{\tilde{h}}{\tilde{a}} \Delta u  \right].
    \end{aligned}    
    \right.
\end{align}
Applying theory of linear systems with constant coefficients.
The numerical flux is
\begin{align}\label{eq:Roe_flux}
    \mathbf{F}_{i+\frac{1}{2}} = \frac{1}{2} \left( \mathbf{F}_L + \mathbf{F}_R \right) - \frac{1}{2} \sum_{j=1}^3 \tilde{\alpha}_j \left| \tilde{\lambda}_j \right| \tilde{\mathbf{R}}^{(j)}.
\end{align}
The Roe flux~\eqref{eq:Roe_flux} is used in the explicit conservative scheme to solve the SWE in 1D.

Entropy fix?



\subsection{Implementation of the FVM in 1D}
The code to solve the SWE in 1D using FVM is based on the Godunov scheme with the exact Riemann solver.
The exact solution of the Riemann problem is found by using the Riemann invariants and the Rankine-Hugoniot conditions~\cite{trento_course}.

The true solution is found by solving the Riemann problem exact, with 5000 cells, and distinguishing between the wetbed or drybed case, and also identifying the shock and rarefaction waves.


\subsection{Godunov's Method}
We consider the Godunov Upwind method, which is a first-order accurate metod to solve non-linear systems of hyperbolic conservation laws~\cite{Toro2024}.
Godunov's method is a fundamental starting point.
In the method we solve the non-linear Riemann problem at each cell interface. 

Consider the initial-boundary value problem (IBVP) for a system of $N$ nonlinear hyperbolic conservation (balance?) laws     
\begin{equation}\label{eq:IBVP_system}
    \begin{cases}
    \text{PDEs: }    &\mathbf{U}_t + \mathbf{F(U)}_x = \mathbf{S(U)}, \quad x \in [a, b], \quad t > 0, \\
    \text{ICs: }    &\mathbf{U}(x,0) = \mathbf{U}^{(0)}(x), \quad x \in [a,b], \\
    \text{BCs: }    &\mathbf{U}(a,t) = \mathbf{B}_{L}(t), \quad \mathbf{U}(b,t) = \mathbf{B}_{R}(t), \quad t \geq 0.
    \end{cases}
\end{equation}
The vectors $\mathbf{B}_L (t)$ and $\mathbf{B}_R (t)$ denote the boundary conditions at the left and right boundaries, respectively.
The Godunov Upwind method in conservative form~\eqref{eq:explicit_conservative_1D_SWE} solves the IBVP~\eqref{eq:IBVP_system}.


\subsection{Implementation of the FVM in 2D}


\subsection{Data generation}










