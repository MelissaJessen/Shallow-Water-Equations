\chapter{Conclusion}\label{ch:conclusion}
In this thesis, we have derived the shallow water equations, and explored the underlying theory and assumptions behind the equations.
We have researched and implemented the FVM for solving the SWE in 1D and 2D, and validated the implementation against known test cases.
We extended the FVM approach to solve the 1D LSWE on a sphere, employing the ERK4 time-stepping method to obtain high-accuracy solutions.
The FVM proved to be a reliable method for solving the SWE, offering high accuracy.

We have also investigated data-driven approaches, specifically CNNs and FNOs, to evalutate their potential as more efficient alternatives to the FVM.
Both models were trained to solve the SWE in 1D and 2D scenarios, using the data generated by the FVM.
For the 1D case, both models performed well, with the CNN achieving a lower overall error and shorter training time.
When tested with new initial conditions, the FNO maintained accuracy compared to the original training data, while the CNN experienced a slight drop in precision.
For the 1D LSWE on a sphere, both models again yielded good results, with the CNN outperforming the FNO in terms of accuracy and training time.
However, for the steepest initial condition, the accuracy of the two models was nearly identical.

For the 2D case, the FNO demonstrated a higher accuracy than the CNN, but the CNN was significantly faster.
When evaluating grid transferability, both models maintained accuracy in transitioning solutions from a coarse grid to a fine grid.
However, the FNO outperformed the CNN in terms of the MSE and the MAE.
By training the FNO on a coarse grid and evaluating it on a fine grid, the prediction time was significantly faster than using the FVM on the fine grid. % over 100 times
Regarding long-term predictions, both models produced good results within the first time steps, but as time progressed, the error increased for the CNN model.
The FNO model maintained accuracy for longer time horizons, demonstrating its potential for long-term predictions, an outcome that aligns well with the literature.

The choice of method ultimately depends on the application.
For scenarios requiring precise simulations and where time is not a critical factor, numerical methods remain the most accurate and reliable option.
However, its computational expence and reliance on fine grid resolutions make it less suitable for rapid simulations, such as real-time flood forecasting.
Conversely, data-driven methods have proven highly effective for fast simulations, meeting the growing demand for computationally efficient models across various fields.
Among these, the FNO model has demonstrated particular promise for grid transferability and long-term predictions, making it a practical alternative to the FVM when speed is a priority and accuracy can be slightly compromised.
This is especially valuable in emergency situations, where rapid predictions are crucial for decision-making.

This study highlights the potential of training and saving data-driven models.
Once trained, these models can quickly generate predictions for new initial conditions and grid sizes, offering a flexible and efficient approach to simulation.
Additionally, they can aid in generating more data to a better understanding of water dynamics, ultimately leading to new knowledge and insights.
A natural extension of this work c be to explore hybrid methods that combine the strengths of both numerical and data-driven approaches, offering a more comprehensive and efficient solution to SWE.
This approach could help overcome the limitations of each method, and provide more robust solutions for real-world applications.
Another interesting extension is to implement the FVM to solve the SWE in spherical coordinates on a planetary scale.
This could be used to generate data for trining data-driven models.
The literature suggests promising results for the spherical Fourier neural operator (SFNO) in forecasting applications.

In a world facing an increasing frequency of extreme weather events, the need for rapid and accurate simulations is more important than ever.
The integration of data-driven methods with numerical techniques could play a vital role in addressing these challenges, ensuring better preparedness and response to natural disasters. 


%\begin{itemize}
%    \item By training the FNO on a coarse grid and evaluate on a fine grid, the run time is XX times faster, while maintaining the same accuracy.
%    \item Long-term prediction: the FNO can predict the solution for a longer time than the training time.
%    It was demonstrated that training on a coarse grid was not suitable to facilitate a long-term prediction for a fine grid.
%    \item Long-term prediction: the FNO model outperforms the CNN model for long-term prediction.
%    \item Grid independence: when training on a coarse grid and evaluating on a fine grid, the predictions by the FNO is much better than the predictions by the CNN. Aligning with the theory that FNOs are grid independent.
%    \item Efficiency: CNN is faster than FNO, but FNO is more accurate. Which we prefer depends on the application.
%    \item CNN has shown potential for many of the same abilities as the FNO - especially in the 1D case.
%    \item Some of the theoretical advantages of the FNO are better realized in the 2D case than in the 1D case.
%    \item Which model we choose is depending on the application.
%\end{itemize}
%FNO's ability to generalize to unseen data.