\chapter*{Abstract}
\addcontentsline{toc}{section}{\textbf{Abstract}}

%\begin{abstract}
The increasing frequency of natural disasters such as floods and tsunamis highlights the need for efficient and accurate simulations of water dynamics.
This thesis investigates the shallow water equations (SWE), used to model water flow in rivers, lakes and coastal areas.
The SWE are solved using the finite volume method (FVM), in one and two dimensions, as well as the linearized SWE (LSWE) in one dimension on a sphere.
While the FVM provides accurate results, its computational cost on high-resolution grids poses challenges in time-sensitive scenarios.

\noindent To address this, data-driven methods, including convolutional neural networks (CNNs) and Fourier neural operators (FNOs), are explored as alternatives to numerical methods.
Trained on data generated by the FVM, these models predict water levels and are evaluated based on accuracy, computational efficiency, grid transferability, and their ability to make long-term predictions.

\noindent For the 1D SWE, the CNN achieves lower errors and faster training times than the FNO, while the FNO maintains accuracy with new initial conditions.
On the sphere, the CNN outperforms the FNO in accuracy and training time, though their performance is nearly identical for the steepest initial condition.
Overall, the FVM is the fastest method for 1D cases.

\noindent In 2D scenarios, the FNO demonstrates higher accuracy than the CNN, but the CNN is significantly faster.
Both models show potential for grid transferability, with the FNO obtaining lower errors than the CNN.
While the FVM is faster on small grids, data-driven methods are more efficient for larger grids.
For long-term predictions, both models perform well initially, but the CNN's error increases over time, while the FNO maintains accuracy.

\noindent 
This work demonstrates the potential of integrating data-driven methods like CNNs and FNOs into hydrodynamic modeling frameworks, providing efficient, scalable simulations to enhance disaster preparedness and water resource management.

%\end{abstract}

%This thesis explores the integration of numerical and data-driven methodologies to solve the shallow water equations (SWE), a cornerstone in computational fluid dynamics.
%Accurate and efficient simulation of shallow water dynamics is essential for critical applications such as flood forecasting, tsunami modeling, and coastal management.
%Traditional numerical approaches, such as the finite volume method (FVM), solve partial differential equations (PDEs) by discretizing the domain into grids.
%For high-resolution grids, numerical methods provide accurate solutions but are computationally expensive, posing a trade-off between accuracy and efficiency.
%\noindent To address this challenge, this study focuses on solving the SWE in 1D, 2D, and 1D spherical coordinates while comparing Fourier neural operators (FNOs) and convolutional neural networks (CNNs) as data-driven approaches.
%The generated data from the FVM is used to train these models.
%FNOs learn mappings between function spaces, enabling grid-independent solution transfer and efficient handling of non-linearities.
%This allows FNOs to achieve zero-shot super-resolution by transferring solutions seamlessly between coarse and fine meshes.
%In contrast, CNNs operate on fixed grids, offering a complementary perspective on spatially localized features.
%\noindent The thesis evaluates the performance of FNOs and CNNs in predicting the evolution of water levels and assesses their computational efficiency and accuracy.
%By training on coarse grids and inferring solutions on finer grids, both approaches demonstrate potential scalability and adaptability for real-time applications.
%The results highlight the strengths and limitations of each method, with FNOs excelling in grid-independent operations and CNNs providing robust localized feature extraction.
%\noindent This work contributes a comprehensive computational toolkit for solving SWE, combining traditional numerical techniques with advanced neural network methodologies.
%The findings emphasize the potential of integrating FNOs and CNNs into hydrodynamic modeling frameworks, paving the way for enhanced disaster preparedness and water resource management.

%This thesis investigates the shallow water equations (SWE), used to model water flow in rivers, lakes and coastal areas. Traditional numerical methods, such as the finite volume method (FVM), are implemented to solve the SWE in one and two dimensions, as well as the linearized SWE (LSWE) in one dimension on a sphere.
%For high-resolution grids, numerical methods provide accurate solutions but are computationally expensive, creating a trade-off between accuracy and efficiency.
%While finer grids improve simulation accuracy, in time-sensitive scenarios such as ongoing or expected floods or tsunamis, the simulation run time can be critical.

%\noindent To adress this challenge, data-driven methods are explored as an alternative to numerical methods.
%Convolutional neural networks (CNNs) and Fourier neural operators (FNOs) are trained on data generated by the FVM to predict the evolution of water levels.
%The methods are evaluated on various metrics such as accuracy, computational efficiency, grid transferability, and their ability to make long-term predictions.

%\noindent For the 1D SWE, the CNN achieves lower overall errors and faster training times compared to the FNO, while the FNO excels in maintaining accuracy with new initial conditions.
%For the 1D LSWE on a sphere, the CNN outperforms the FNO in terms of accuracy and training time, though their performance is nearly identical for the steepest initial condition.
%In general, for the 1D cases, the FVM remains the fastest method.

%\noindent In 2D scenarios, the FNO demonstrates higher accuracy than the CNN, but the CNN is significantly faster.
%Both models show potential for grid transferability, with the FNO outperforming the CNN in terms of mean squared error (MSE) and mean absolute error (MAE).
%When analysing run time, the FVM is fastest for smaller grids, but for larger grids, the data-driven methods prove to be more efficient.
%For long-term predictions, both models perform well initially, but the error for the CNN model increases over time, while the FNO model maintains accuracy.

%This work demonstrates the potential of data-driven methods for rapid simulations, particularly in emergency scenarios requiring real-time predictions.
%While traditional numerical methods remian the most reliable for high-accuracy applications, data-driven approaches offer significant advantages in computational efficiency and scalability.
%By training models in advance, data-driven methods can quickly generate predictions for new initial conditions and grid sizes, offering a flexible and efficient approach to hydrodynamic modelling.
%Ultimately, integrating these methods into disaster management strategies could enhance preparedness and response to natural disasters.

%This study evaluates data-driven approaches, specifically convolutional neural networks (CNNs) and Fourier neural operators (FNOs), as efficient alternatives to the finite volume method (FVM) for solving the shallow water equations (SWE).
%For 1D SWE, CNNs achieve lower errors and faster training times than FNOs, while FNOs maintain accuracy with new initial conditions.
%On the sphere, CNNs outperforms FNOs in accuracy training time, though for the steepest initial condition, their performance is nearly identical.
%Overall, the FVM is the fastest method for 1D cases.

%In 2D scenarios, FNOs demonstrate higher accuracy, but CNNs are significantly faster. Both models transfer well across grids, with FNOs outperforming CNNs in terms of mean squared error and mean absolute error.
%While the FVM is faster on small grids, data-driven methods are more efficient for larger grids. For long-term predictions, FNOs maintain accuracy over time, while CNN errors increase.

%This work highlights the potential of data-driven methods for rapid and scalable hydrodynamic simulations, particularly in time-critical applications like disaster management.
%While numerical methods remain superior for high-accuracy tasks, pre-trained models enable efficient and flexible predictions, enhancing disaster preparedness and response.

